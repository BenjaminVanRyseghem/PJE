\documentclass[a4paper,10pt]{article}
\input{/Users/benjamin/Documents/Education/LaTeX/macro.tex}

\title{PJE: TP1}
\author{Benjamin \bsc{Van Ryseghem}}

\begin{document}
\maketitle

\section{Dioptre plan}
\subsection{Question 1}

L'indice de r�fraction du milieu de droite est \emph{plus grand} que l'indice de r�fraction de l'indice de gauche car les rayons se rapprochent de la normale.

\subsection{Question 2}
On utilise la loi:
\begin{math}
	n_1.sin(i_1) = n_2.sin(i_2)
\end{math}

\begin{align*}
	1.sin(45) &= n_2.sin(20)\\
	n_2 &= \frac{sin(20)}{sin(45)}
\end{align*}

\subsection{Question 3}
Le dioptre n'est pas \emph{stygmate} car quand la source est situ� dans le milieu le plus r�fractaire, on observe une r�flexion de certains rayons.

\subsection{Question 4}
L'image est \emph{virtuelle} puisque le point de concourance des rayons images se situe avant le dioptre.

\subsection{Question 5}
Pour un observateur situ� dans le milieu de droite, l'objet semble \emph{plus pr�s} qu'il ne l'est r�ellement. 

\section{R�flexion totale}
\subsection{Question 1}
Les rayons entrant dans la plaque de plexiglass et ceux sortant sont \emph{parall�les}. Il y a par contre une translation du faisceau d�t aux r�fractions successives.

\subsection{Question 2}
On remarque qu'il existe un angle (environ 25�) apr�s lequel les rayons se retrouvent "prisonniers" de la plaque de plexiglass. Il n'y a plus r�fraction mais r�flexion.

\subsection{Question 3}
L'angle vaut environ 25� ($arctan(0.48)$).

\begin{align*}
	n_1.sin(25) &= 1.sin(90)\\
	n_1 &= \frac{sin(90)}{sin(25)}\\
\end{align*}

\subsection{Question 4}
\begin{align*}
	1.sin(\frac{\Pi}{2}-i_1) &= 1.51.sin(\frac{\Pi}{2})\\
	sin(\frac{\Pi}{2}-n_1) &= 1.51\\
	\frac{\Pi}{2}-n_1 & = arcsin(1.51)\\
	n_1 & = \frac{\Pi}{2}-arcsin(1.51)
\end{align*}

Le demi angle est donc de $\frac{\frac{\Pi}{2}-arcsin(1.51)}{2}$.

\section{Objectif et mise au point}
\subsection{Question 1}

L'image est nette quand le foyer image se trouve dans le plan du capteur.

\subsection{Question 2}
La mise au point est conserv� si la projection orthogonale de la source sur l'axe optique reste constante.



\signature[Benjamin Van Ryseghem \\ Fran�ois Lepan]

\end{document}