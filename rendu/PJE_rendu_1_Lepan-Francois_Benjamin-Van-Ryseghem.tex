 \documentclass[a4paper,10pt]{article}
\input{/Users/WannaGetHigh/Documents/workspace/latex/macros.tex}

\title{Rapport de PJE}
\author{Fran�ois \bsc{Lepan},Benjamin \bsc{Van Ryseghem}}

\begin{document}
\maketitle

\section{Acquisition Video}

Dans cette partie nous avons appris � utiliser et r�gler la cam�ra afin de comprendre les limites et les flux de donn�es en provenance de la camera.

Tout d'abord nous avons effectu� des tests sur la cadence d'acquisition de la cam�ra. En faisant varier l'horloge pixel on se rend compte que la cadence d'acquisition est directement li� � celle-ci. Si on prend une horloge � 40 MHz on a une cadence maximal de 87 images/seconde, et si on r�duit l'horloge a 10 MHz on r�duit alors la cadence a 22 images/seconde. En effet si on envoie 4 fois moins d'informations dans le tube on est obliger d'acqu�rir 4 fois moins d'images depuis la camera.

Ensuite nous nous sommes int�resser au temps d'acquisition. Apr�s quelques tests nous avons remarqu� que plus le temps d'acquisition est long plus les images sont claires.

\section{Taille de l'image}

Ici nous nous sommes int�ress� � la taille de l'image en changeant certain param�tre. La taille de l'image est issue des dimensions du capteur. 
Nous avons aussi changer la taille de l'image acquise et la position de celle-ci par rapport aux capteur de la cam�ra et nous nous sommes rendu compte que tant que l'on s�lectionnais une taille inf�rieur aux dimensions maximal tout en restant dans les bornes du capteur nous pouvions s�lectionner une image au pixel pr�s.

??

\section{image N&B et couleur}

???

\section{OpenCV : lecture et affichage d'une}

???

\section{Exercice }

\end{document}